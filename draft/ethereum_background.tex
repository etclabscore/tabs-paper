%! Author = ia
%! Date = 4/22/22

% Preamble
\documentclass[11pt]{article}

% Packages
\usepackage{amsmath}

% Document
\begin{document}

%% %%%%%%%%%%%%%%%%%%%%%%%%%%%%%%%%%%%%%%%%%%%%%%%%%%%%%%%%%%%%%%%%%%%%%%%%%%%%%%%%%%%%%%%%%%%%%%%%%%%%%%%%%%

	\pagebreak
	\section{\normalsize{Appendix: Ethereum Background}}\label{sec: appendix}

	The parameters and logic of this paper's proposal take the existence of a
	system equivalent with that of Ethereum's at the time of writing.
	We strongly suggest that the Ethereum Yellow Paper\nolinebreak
	\footnote{\url{https://github.com/ethereum/yellowpaper/tree/fabef2531a8f8e772a4bf5be23191044d0ce3569}}
	be read and understood as a precondition for interpretation and evaluation
	of the work of this project.
%should we state the 'Berlin' version? ISAAC: Done.
	Important and referenced concepts of Ethereum's protocol are discussed in this
	section for the the informational value of redundancy.

	\subsection{\small{Ethash}}

	Ethereum's Proof-of-Work protocol "Ethash" governs block emissions. Solutions
	to a constantly varying guessing game require time to discover -- via trial and
	error -- and are a notably required field in block headers. Authors of blocks
	providing valid solutions to these puzzles are called miners; mining is the
	process of searching for puzzle solutions. \\
	\\
%should we define "block emissions" for the non-expert?
	A very difficult puzzle (probably requiring many guesses) is expected to take
	more time to solve than an easier puzzle (requiring fewer guesses).

	\subsection{\small{Difficulty}}
	Ethereum regulates its block emission rate using a header value called
	$\mathrm{Difficulty}$ ($H_d$). This value is used by the Ethash protocol as a
	parameter for puzzle solution validation. Generally speaking, the Difficulty
	value for some block can be thought of as the number of wrong guesses
	("hashes") expected before a valid puzzle solution for a child block is
	found.\nolinebreak
	\footnote{Formally, difficulty is governed by the relation of a fixed size
	bit-field (called the 'nonce') in the domain of the block hash, into which a
	miner writes its guess, and a fixed size bit-field in the output range of the
	hash function, which takes on values induced by the miner's guess. The
	difficulty is measured by the number of required zero entries in the output
	bit-field, which defines the feasible output range.  The size of the set of
	guesses that map into the feasible range expands and contracts with the size of
	the feasible range. For a given rate of guesses per unit of time, the
	difficulty induces a cumulative distribution function of the probability of a
	solution being found by each point in time. From this an expected interval
	between timestamps can be derived.}
	An adjustment algorithm governs the rise and fall of this value, accepting a
	block timestamp interval and parent difficulty as parameters;
	this provides a feedback loop joining block emission rates (via sequential
	block timestamps and thus relative intervals) with puzzle difficulty.
	Given a target block interval, difficulty can be adjusted dynamically,
	incrementally, such that the difficulty value will cause blocks to be authorable
	at rates approaching the target rate.
	In Ethereum the parameters are tuned to produce a median 9 seconds, or about 14
	seconds on average.
	Network block emission rates are typically modeled with a Poisson
	distribution.\nolinebreak
	\footnote{\url{https://ethresear.ch/t/deep-dive-into-current-pow-difficulty-adjustment-algorithm-and-a-possible-alternative/5267}}\nolinebreak
	\footnote{\url{https://blog.ethereum.org/2015/09/14/on-slow-and-fast-block-times/}}\nolinebreak
	\footnote{\url{https://en.wikipedia.org/wiki/Proof_of_work#Variants}}\nolinebreak
	\footnote{\url{https://arxiv.org/pdf/1901.04620, Section III.A}}

	\textit{e.g.} Given a block difficulty $H_d$ = 63115 and a network target
	emission interval of 13 seconds, we deduce that the modeled network average
	hashrate is 63115 hashes / 13 seconds = 4855 hashes/second. A block generated
	in 3 seconds should cause the difficulty to rise; a block generated in 30
	seconds should cause it to fall; making the next block respectively harder or
	easier (slower or faster) to author.

	\subsection{\small{Canonical Arbitration\footnote{\url{https://github.com/ethereum/go-ethereum/blob/f0328f241b7c3def217b0c2dce1a1b297f979a37/core/forkchoice.go#L77}}}}

	The "Total Difficulty" value for any chain segment is the sum of the $H_d$
	values of block headers in the segment.
	The canon-arbitration algorithm used today by Ethereum defines
	that preference is given with priority to any valid subtree having the greatest "Total Difficulty" value.
	In the case of equivalent TD values, segments are preferred having lesser latest block numbers.
	In the case of equivalent TD and block height (number) values, if a node acts
	on behalf of the block's registered author, that block is preferred.
	If there is no authorship beneficiary interest (as in a non-mining node, or
	that of a non-winning miner), a figurative coin is tossed, per the protocol
	described by Eyal and Sirer.\nolinebreak
	\footnote{\url{http://www.cs.cornell.edu/~ie53/publications/btcProcFC.pdf}}

	\subsection{\small{Chain ID}}

	As specified by EIP155\nolinebreak
	\footnote{\url{https://eips.ethereum.org/EIPS/eip-155}} a feature called
	\textit{Chain ID} was introduced on the Ethereum network (ETH) at block
	2675000, and on the Ethereum Classic network (ETC) at block 3000000.

	This feature, generally considered, allows transactions to be made exclusive to
	certain predefined segments, or entire chains, by validating a match between a
	transaction-specified \textit{Chain ID} field value and that of a hardcoded
	(constant) network protocol configuration.
	The mechanism ascribes an arbitrary positive integer to a network configuration
	and transactions use this value when signing a transaction intended for that
	subset of chain state.
	Transactions are not required to specify a \textit{Chain ID} value for
	legacy-compatibility reasons.

	This feature was introduced after "The DAO Fork" (at block 1920000) caused
	Ethereum to become both Ethereum and Ethereum Classic by the network's partial rejection of an arbitrary chain state mutation.
	Until the introduction of Chain ID, transactions after this hardfork had no way
	of being exclusive to one of either of these two chains.
	Between blocks 1920000 and 2675000on ETH, transactions were ambiguously valid on either, or both.
	Today, Chain ID transactions specifying Chain ID \texttt{1} are only eligible
	on ETH, and those with Chain ID \texttt{61} are only eligible on ETC, per the
	respective configurations eventually introduced on both networks with their
	independent implementations of the EIP155 feature.


\end{document}